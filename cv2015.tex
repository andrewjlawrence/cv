\documentclass[11pt,a4paper,sans]{moderncv}

%% ModernCV themes
\moderncvstyle{casual}
\moderncvcolor{blue}
\renewcommand{\familydefault}{\sfdefault}
\nopagenumbers{}

%% Character encoding
\usepackage[utf8]{inputenc}

%% Adjust the page margins
\usepackage[scale=0.75]{geometry}
\usepackage{comment}
%% Personal data
\firstname{Andrew}
\familyname{Lawrence}
\title{Senior Systems Software Engineer}
\address{Flat 1, 69 Victoria Road, Parkstone, Poole}{BH12 3AA}
\mobile{(+44) (0)7557342986}
\phone{(+44) (0)1792 205678 ext. 4534}
\fax{(+44) (0)1792 295708}
\email{ajlawrence@acm.org}
\homepage{www.safesecuresystems.co.uk/}
%%\extrainfo{additional information}

%%\quote{Some quote (optional)}

%%------------------------------------------------------------------------------
%% Content
%%------------------------------------------------------------------------------
\begin{document}
\makecvtitle

\section{Personal Statement}
6 years experience of researching, In cooperation with Siemens (Rail Automation UK), the modelling and verification of train control systems for which I obtained a Masters and am working towards a Doctorate in theoretical computer science under the supervision of Monika Seisenberger. As an undergraduate I obtained a $1^{st}$ class degree. I am a prolific public speaker and have given many talks infront of both academic and industrial audiences. I enjoy working with computers and constantly seek to increase both my breadth and depth of knowledge with a view to becoming an expert in my chosen field. I have recently been working in a software systems engineering role for Hitachi Data Systems, to which end, I have designed and implemented a new dynamic distributed role based access control system in C++ for a data protection/disaster recovery product.
%%%\section{Current Research Interests}


%%%\section{Career Ambitions}
%%%To become a successful Business Analyst with a long proven track record of working on challenging %%%and technically difficult projects. Eventually, I see myself taking on more of a project management %%%role using my vision to guide others.
\section{Technical Skills}
\begin{center}
\includegraphics[scale = 0.5]{wordle2015}
\end{center}
\section{Experience}
\cventry{September 2014 - Present}{Senior Systems Software Engineer}{Hitachi Data Systems}{}{}{Designed and implemented a dynamic distributed role based access control system and a RESTful file system management/file transfer utility. Technologies and techniques used: C++, UML, Enterprise Architect, OpenLDAP, SLAPD, SASL, Active Directory, Windows Server, Linux, WireShark, Boost, QT, unit testing, Visual Studio, Bash, MongoDB, multi-threading.}
\cventry{2009-2015}{Researcher}{Swansea University}{}{}{Researched the following topics:
\begin{enumerate} \item Formal modelling and verification of the European Rail Traffic Management System.
\item Extracting verified decision procedures from constructive proofs.
\item Formal verification of solid state interlocking programs.
\end{enumerate}
}
\cventry{2009 - 2014}{Teaching Assistant}{Swansea University}{}{}{As a postgraduate, undertaking lab classes, marking coursework and giving revision lectures.}
\section{Education}
%
\cventry{2011- 2015} {PhD Computer Science}{Swansea University}{}{}{}  
\cventry{2012}{NATO Science for Peace and Security, Advanced Study Institute Summer School}{Marktoberdorf}{"Engineering Dependable Software Systems"}{}{I attended lectures on a variety of formal methods including "Model Checking and Synthesis", "Abstraction, Refinement and Decomposition for Systems Engineering" and "How to Verify Your Software?" delivered by world leading computer scientists.}
%
%The current working title of my thesis is "Application of Automatic and Interactive Theorem Proving to the Development and Verification of a Combined Railway Control System". My PhD involves collaboration on an industrial project which aims to combine interactive
%theorem proving and model checking to verify railway control systems.  \\
%Firstly, my work involved program extraction from a formal proof to obtain  verified DPLL SAT-solving
%algorithms. The formalisation has been carried out in the Minlog system  and the algorithms
%have been extracted as both Minlog terms and Haskell code. This is a case study into the
%application of program extraction and provides a new approach for the formal development and
%verification of decision procedures. It also shows that efficiency considerations can be taken into
%account at the proof level rather than at the level of code where they may lead
%to more complicated verification tasks. These verified SAT solving algorithms have so far been applied to verify ladder logic programs describing railway interlockings. \\
%Secondly, I have been working on the specification and verification of the European Rail Traffic Management System (ERTMS). This is a complicated hybrid system containing both discrete and continuous data and processes. My approach models ERTMS as a hybrid automata before  formalising it using Real Time Maude. \\
%I have regularly given talks throughout my time in academia at both conferences and internally at seminars. 
\cventry{2009--2010}{MRes Logic and Computation}{Swansea University}{}{}{}  % arguments 3 to 6 can be left empty
%My thesis was entitled "Verification of Railway Interlockings in SCADE". It contained two approaches. In the first approach I modified an existing tool to translate ladder logic programs into the Scade Language. This tool was applied to railway interlocking programs written in ladder logic. I then used the built in model checking capabilities of the Scade suite to verify these interlocking programs. The second approach involved modelling and capturing the behaviour of segments of railway in a modular fashion. I then verified properties of the topology and the movement of trains.
%
%
\cventry{2006--2009}{BSc (Hons) Computer Science. 1st Class}{Swansea University}{}{}{}
\cventry{2003-2005}{A Levels: Mathetmatics (A) Physics (C). AS Levels: Computing (B) Chemistry (C) Biology(D)}{New College Swindon and Faringdon Community College}{}{}{}
%
%My dissertation was entitled "The Formalisation and Verification of RSA PSS". I formalised and verified part of the security protocol RSA-PSS in the Minlog System.  I also verified the Needham-Schroeder protocol as part of the "High Integrity Systems" course.}  % arguments 3 to 6 can be left empty
%%\cvitem{title}{ \emph{Title} }
%%%\subsection{Languages}
%%%\cvitemwithcomment{Mandarin}{Basic}{I attended a beginner's mandarin course at Swansea %%%university.}
%\cvitemwithcomment{French}{Basic}{I studied for a GCSE in French.}
%%%\cvdoubleitem{category X}{XXX, YYY, ZZZ}{category Y}{XXX, YYY, ZZZ}
%
%%%\cvlistdoubleitem{Item 2}{Item 3}
%% ...
%\nocite{*}
\section{Achievements}
Being elected as college representative for over one hundred postgraduate research students. In this role I sat on several decision making committees and was part of a team pursuing an Athena SWAN Bronze award for recruitment, retention and promotion of women in STEMM in higher education. 

This year I decided to take up running and after 3 months of intensive training ran my first ever half marathon.

I was part of a kung fu demonstration and lion dance, which was watched and enjoyed by over 200 people,  as part of the Swansea Chinese new year celebrations in 2012 and 2013.
%

%
%
%\cventry{2012- 2013}{College Representative}{Swansea University}{}{}{I was elected to represent over 100 postgraduate research (PGR) students in the College of Science. I sit on several decision making committees at the college level. My manifesto was entitled "Resources for Research". It contained the following main points: \begin{itemize} \item Providing every (PGR) student with a professional looking staff email address rather than one consisting of their student number.
%\item Increase provision for travel expenses for (PGR) students .
%\item Implement a system for book requests for (PGR) students.
%\end{itemize}
%The implementation of the first point is currently in progress. A new system has been designed for the whole university that allows for students to request books through their college and subject representatives.}%


\section{Interests and Hobbies}
Keeping fit through yoga, running and martial arts. In recent years I have developed a interest in politics and campaigned door to door for a major party during the 2010 general election. I enjoy philosophy in particular that of east Asia and have read works by Lao Tzu, Chang Tzu and Musashi.

%\cventry{}{Chinese Martial Arts}{}{}{}{I have been practising martial arts since I was a child training with many different teachers. I currently travel to Cardiff every week to train a rare style of kung fu with a Chinese grandmaster. I am part of my kung fu schools lion dance team and have performed at several Chinese new year celebrations. I was part of a lion dance team and kung fu demonstration for the 2012  Chinese new year celebrations in Swansea performing in front of the mayor and several hundred people.}
%\cventry{}{General Exercise}{}{}{}{I regularly run and attend the gym. I particularly focus on functional strength training such as powerlifting and kettlebell exercises.}
%\cventry{}{Live Music}{}{}{}{I have played guitar and bass in a band and given several live performances in bars and pubs. I also enjoy attending festivals and concerts.}



\section{References}
References available on request.
\begin{comment}
\cvdoubleitem{Reference 1}{Dr Monika Seisenberger, \newline{}Department of Computer Science,\newline{} Swansea University,\newline{}
Singleton Park,\newline{} Swansea,  SA2 8PP, \newline{} UK, \newline{} M.Seisenberger@Swansea.ac.uk \newline Tel: (+44) (0)1792 602131}{Reference 2}{Professor Faron Moller, \newline{}Department of Computer Science,\newline{} Swansea University,\newline{}
Singleton Park,\newline{} Swansea,  SA2 8PP, \newline{} UK, \newline{} F.G.Moller@Swansea.ac.uk \newline Tel: (+44) (0)1792 295160}
\end{comment}
\nocite{*}
\medskip
\bibliographystyle{plain}
\bibliography{lit}               


\end{document}
